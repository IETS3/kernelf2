%
% The first command in your LaTeX source must be the \documentclass command.
\documentclass[sigplan,screen]{acmart}

\usepackage{amsmath}
\usepackage{amssymb}
\usepackage{hyperref} 
\usepackage{soul}  
\usepackage{color}
\usepackage{xcolor}
\usepackage{graphicx}
\usepackage{inputenc} 
\usepackage{comment}
%\usepackage{wrapfig} 
\usepackage{ae}  
\usepackage{sidecap}
\usepackage{url}
\usepackage[T1]{fontenc} 
\usepackage{verbatim}
\usepackage{listings} 
\usepackage{tikz}
\usepackage{etoolbox}
\usepackage{mdframed}
\usepackage{multirow}
\usepackage{setspace}
\usepackage{balance}
\usepackage{lastpage}

\newenvironment{code}
    {\noindent
     \rule{\columnwidth}{0.6pt}
     \vspace{-4mm}
     \scriptsize \verbatim}
    {\endverbatim \normalsize
     \vspace{-4mm} 
    \rule{\columnwidth}{0.6pt} 
 	}
\usepackage{wrapfig}
\usepackage{relsize}
\usepackage{tabularx}
\usepackage{microtype}
\usepackage{multirow}
\usepackage{semantic}

\usepackage{booktabs}
\usepackage{paralist}
% important for nicer code font
\usepackage[scaled]{beramono}


\makeatletter
    \def\balanceissued{unbalanced}%flag to indicate if \balance has been used
    \let\oldbibitem\bibitem
    \def\bibitem{%
        \ifnum\thepage=\lastpage@lastpage%
            \expandafter\ifx\expandafter\relax\balanceissued\relax\else%
                \balance%
                \gdef\balanceissued{\relax}\fi%
            \else\fi%
        \oldbibitem}
\makeatother

\definecolor{lightlightgray}{gray}{0.9}
\definecolor{codecolor}{gray}{0.15} 
\definecolor{javadocblue}{rgb}{0,0,0.8} % javadoc
\definecolor{blackblack}{rgb}{0,0,0}
  
\definecolor{myGray1}{rgb}{0.85,0.85,.85}

\newif\iflongpaper
\longpaperfalse

\lstset{
basicstyle=\ttfamily\scriptsize\color{codecolor}, 
keywordstyle=\color{blackblack}\bfseries,       
% keywordstyle=\color{javadocblue},     
commentstyle=\color{gray},              
numbers=none,                           
numberstyle=\tiny,                      % Line-numbers fonts
stepnumber=1,                           % Step between two line-numbers
numbersep=5pt,                          % How far are line-numbers from code
%backgroundcolor=\color{lightlightgray}, % Choose background color
tabsize=2,     
frame=single,
%xleftmargin=1.5mm,
%xrightmargin=1.5mm,
captionpos=b,   
keepspaces=true,
xleftmargin=1.5mm,
xrightmargin=1.5mm,                        % Caption-position = bottom
breaklines=true,    
escapeinside={@}{@},
                    % Automatic line breaking?
breakatwhitespace=false,                % Automatic breaks only at whitespace?
showspaces=false,                       % Dont make spaces visible
showtabs=false,                         % Dont make tabls visible
columns=flexible,                       % Column format
morekeywords={}
} 

\lstdefinelanguage{kernelf}
{keywords={val, string, real, boolean, int, true, false, if, then, else,
option, some, none, with, fun, alt, success, try, error, list, ext, set,
map, collection, type, bind, exec, value, it, where, any, all, record,
check, if, otherwise, attempt, as, old, enum, qualified, check,
assert, equals, test, case, constraint, failure, number, precision, round, up,
cut, to, down, limit, cast, R, RM, M, box, val, update, newtx, intx, of, reduce, in, month, TT,
calculation, depends, for, allInCurrentYear},
%otherkeywords={->,=>,[,],!,\,},
otherkeywords={!,\,},
alsodigit={0,1,2,3,4,5,6,7,8,9,0},
stringstyle=\color{OliveGreen},
numbers=none,
mathescape=true,
showstringspaces=false,
sensitive=true,
tabsize=2,
columns=fullflexible,
morestring=[b]",
comment=[l]{//},
basicstyle=\scriptsize\ttfamily\color{darkgray},
keywordstyle=\bf\scriptsize\ttfamily\color{black},
}

\newmdenv[
  topline=true,
  bottomline=true,
  rightline=true,
  leftline=true,
  skipabove=0mm,
  linewidth=1.2pt,  
  linecolor=black,  
  skipbelow=0mm,
  leftmargin=0pt,
  rightmargin=0pt,
  innertopmargin=5pt,
  innerleftmargin=3pt,
  innerrightmargin=3pt,
  innerbottommargin=3pt
  ]
{mvbox}

%\newcommand\citeX[1]{cite{{#1}}}
%\newcommand\itemis{itemis}
%\newcommand\anon[2]{{#1}}


%
% defining the \BibTeX command - from Oren Patashnik's original BibTeX documentation.
\def\BibTeX{{\rm B\kern-.05em{\sc i\kern-.025em b}\kern-.08emT\kern-.1667em\lower.7ex\hbox{E}\kern-.125emX}}
    
%%% If you see 'ACMUNKNOWN' in the 'setcopyright' statement below,
%%% please first submit your publishing-rights agreement with ACM (follow link on submission page).
%%% Then please update our instructions page and copy-and-paste the NEW commands into your article.
%%% Please contact us in case of questions; allow up to 10 min for the system to propagate the information.
%%%
%%% The following is specific to SLE '19 and the paper
%%% 'Shadow Models: Incremental Transformations for MPS'
%%% by Markus Voelter, Klaus Birken, Sascha Lisson, and Alexander Rimer.
%%%
\setcopyright{ACMUNKNOWN}
\acmPrice{}
\acmDOI{10.1145/3357766.3359528}
\acmYear{2019}
\copyrightyear{2019}
\acmISBN{978-1-4503-6981-7/19/10}
\acmConference[SLE '19]{Proceedings of the 12th ACM SIGPLAN International Conference on Software Language Engineering}{October 20--22, 2019}{Athens, Greece}
\acmBooktitle{Proceedings of the 12th ACM SIGPLAN International Conference on Software Language Engineering (SLE '19), October 20--22, 2019, Athens, Greece}

%
% These commands are for a JOURNAL article.
%\setcopyright{acmcopyright}
%\acmJournal{TOG}
%\acmYear{2018}\acmVolume{37}\acmNumber{4}\acmArticle{111}\acmMonth{8}
%\acmDOI{10.1145/1122445.1122456}

%
% Submission ID. 
% Use this when submitting an article to a sponsored event. You'll receive a unique submission ID from the organizers
% of the event, and this ID should be used as the parameter to this command.
%\acmSubmissionID{123-A56-BU3}

%
% The majority of ACM publications use numbered citations and references. If you are preparing content for an event
% sponsored by ACM SIGGRAPH, you must use the "author year" style of citations and references. Uncommenting
% the next command will enable that style.
%\citestyle{acmauthoryear}

%
% end of the preamble, start of the body of the document source.
\begin{document}

%
% The "title" command has an optional parameter, allowing the author to define a "short title" to be used in page headers.
\title{Shadow Models: Incremental Transformations for MPS}

%
% The "author" command and its associated commands are used to define the authors and their affiliations.
% Of note is the shared affiliation of the first two authors, and the "authornote" and "authornotemark" commands
% used to denote shared contribution to the research.
\author{Markus Voelter}
\affiliation{%
  \institution{independent/itemis}
  \streetaddress{Oetztaler Strasse 38}
  \city{Stuttgart}
  \postcode{70727}
  \country{Germany}
}
\email{voelter@acm.org}

\author{Klaus Birken}
\affiliation{%
  \institution{itemis AG}
  \streetaddress{Industriestrasse 6}
  \city{Stuttgart}
  \postcode{70565}
  \country{Germany}
}
\email{birken@itemis.de}

\author{Sascha Lisson}
\affiliation{%
  \institution{itemis AG}
  \streetaddress{Industriestrasse 6}
  \city{Stuttgart}
  \postcode{70565}
  \country{Germany}
}
\email{lisson@itemis.de}


\author{Alexander Rimer}
\affiliation{%
  \institution{itemis AG}
  \streetaddress{Industriestrasse 6}
  \city{Stuttgart}
  \postcode{70565}
  \country{Germany}
}
\email{rimer@itemis.de}


%\author{Klaus Birken}
%\affiliation{%
%  \institution{itemis AG}
%  \streetaddress{Industriestrasse 6}
%  \city{Stuttgart}
%  \postcode{70565}
%  \country{Germany}
%}
%\email{birken@itemis.de}

%\author{Alexander Rimer}
%\affiliation{%
%  \institution{itemis AG}
%  \streetaddress{Industriestrasse 6}
%  \city{Stuttgart}
%  \postcode{70565}
%  \country{Germany}
%}
%\email{rimer@itemis.de}

%\author{Tamas Szabo}
%\affiliation{%
%  \institution{itemis AG}
%  \streetaddress{Industriestrasse 6}
%  \city{Stuttgart}
%  \postcode{70565}
%  \country{Germany}
%}
%\email{szabo@itemis.de}





%
% By default, the full list of authors will be used in the page headers. Often, this list is too long, and will overlap
% other information printed in the page headers. This command allows the author to define a more concise list
% of authors' names for this purpose.
\renewcommand{\shortauthors}{Voelter et al.}

% The abstract is a short summary of the work to be presented in the article.
\begin{abstract}
Shadow Models is an incremental transformation framework for MPS. The name is motivated
by the realization that many analyses are easier to do on an model whose structure is different from
what the user edits. To be able to run such analyses interactively in an IDE, these ``shadows''
of the user-facing model must be maintained in realtime, and incrementality can deliver the needed
short response times. Shadow Models is an incremental model transformation engine for MPS. 
In the paper we motivate the system through example use cases, 
and outline the transformation framework.
\end{abstract}

%
% The code below is generated by the tool at http://dl.acm.org/ccs.cfm.
% Please copy and paste the code instead of the example below.
%


\begin{CCSXML}
<ccs2012>
<concept>
<concept_id>10011007.10011006.10011066.10011070</concept_id>
<concept_desc>Software and its engineering~Application specific development environments</concept_desc>
<concept_significance>500</concept_significance>
</concept>
<concept>
<concept_id>10011007.10011006.10011050.10011017</concept_id>
<concept_desc>Software and its engineering~Domain specific languages</concept_desc>
<concept_significance>300</concept_significance>
</concept>
</ccs2012>
\end{CCSXML}

\ccsdesc[500]{Software and its engineering~Application specific development environments}
\ccsdesc[300]{Software and its engineering~Domain specific languages}

%
% Keywords. The author(s) should pick words that accurately describe the work being
% presented. Separate the keywords with commas.
\keywords{domain-specific languages, model transformations, incrementality, language workbenches, MPS}

\newcommand\parhead[1]{\vspace{1mm}\noindent\textbf{{#1}}\ \ }  
\newcommand\parheadX[1]{\vspace{-0.0mm}\noindent\textbf{{#1}}\ \ }  
\newcommand{\toolurl}[1]{\footnote{\ic{#1}}}
\newcommand{\ic}[1]{\changefont{cmtt}{m}{n}{#1}\normalfont}  % inline code
\newcommand{\changefont}[3]{\fontfamily{#1}\fontseries{#2}\fontshape{#3}\selectfont}
\newcommand{\fig}[1]{Fig. \ref{#1}}  % inline code
\newcommand{\sect}[1]{Section \ref{#1}}  % inline code

\newcommand\TODO[1]{\vspace{1mm}\noindent\textbf{\color{red} {{TODO: {#1}} }}}  
\newcommand\MARKUS[1]{\vspace{1mm}\noindent\textbf{\color{blue} {{MARKUS: {#1}} }}}  
\newcommand\SASCHA[1]{\vspace{1mm}\noindent\textbf{\color{blue} {{SASCHA: {#1}} }}}  
\newcommand\MICHAEL[1]{\vspace{1mm}\noindent\textbf{\color{blue} {{FRAGE MICHAEL: {#1}} }}}
\newcommand\SERGEJ[1]{\vspace{1mm}\noindent\textbf{\color{blue} {{SERGEJ: {#1}} }}}
\newcommand\ANNA[1]{\vspace{1mm}\noindent\textbf{\color{blue} {{ANNA: {#1}} }}}
\newcommand\KLAUS[1]{\vspace{1mm}\noindent\textbf{\color{blue} {{KLAUS: {#1}} }}}
\newcommand\DATEV[1]{\vspace{1mm}\noindent\textbf{\color{teal} {{FRAGE DATEV: {#1}} }}}  

\newcommand\mvlightbox[1]{{\setlength{\fboxsep}{2pt}\colorbox{gray}{\textcolor{white}{\textsf{\textbf{\footnotesize {#1}}}}}}}  
\newcommand\mvdarkbox[1]{{\setlength{\fboxsep}{2pt}\colorbox{darkgray}{\textcolor{white}{\textsf{\textbf{\footnotesize {#1}}}}}}}  

%\newcommand\COneAccessibility{\colorbox{gray}{\textcolor{white}{\textsf{\textbf{\footnotesize C1}}}}}  
\newcommand\COneExtEvo{\mvlightbox{C1}}  
\newcommand\CTwoBusAgil{\mvlightbox{C2}}  
\newcommand\CThreeDeploy{\mvlightbox{C3}}  

\newcommand\RQOneComplexity{\mvdarkbox{RQ1}}  
\newcommand\RQTwoQuality{\mvdarkbox{RQ2}}  
\newcommand\RQThreeTeaching{\mvdarkbox{RQ3}}  
\newcommand\RQFourITProcess{\mvdarkbox{RQ4}}  

\newcommand\OldSys[1]{\vspace{1mm} \begin{spacing}{0.9} \noindent \colorbox{myGray1}{\textcolor{black}{\textsf{\textbf{\footnotesize OLD}}}} \small {#1} $\blacksquare$ \normalsize \end{spacing} }  
\newcommand\OldSysMeta{\colorbox{myGray1}{\textcolor{black}{\textsf{\textbf{\footnotesize OLD}}}}}  

%
% This command processes the author and affiliation and title information and builds
% the first part of the formatted document.
\maketitle 

 

\input{01-intro.ltx}
\input{02-examples.ltx}
\input{03-engine.ltx}
\input{04-related.ltx}
\input{05-future.ltx} 
\input{06-conclusions.ltx}












\bibliographystyle{abbrv}
\bibliography{doc}





\end{document}
